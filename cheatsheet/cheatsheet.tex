% !TeX TS-program = xelatex

\documentclass[10pt,landscape,a4paper]{article}

\usepackage[framemethod=TikZ]{mdframed}
\mdfsetup{skipabove=0pt,skipbelow=0pt}

\usepackage{multicol}
\usepackage{colortbl}
\usepackage{array, multirow}

\usepackage[right=2.5mm, left=2.5mm, top=2.5mm, bottom=2.5mm]{geometry}
\usepackage{tikz}
\usepackage{graphicx}
\usepackage[percent]{overpic}
\graphicspath{{./images}}
\usepackage[export]{adjustbox}

\usepackage[framemethod=TikZ]{mdframed}
\mdfsetup{skipabove=0pt,skipbelow=0pt}
\usepackage{menukeys}

\usepackage{xcolor}
\colorlet{citecolor}{black}
\colorlet{linkcolor}{black}
\colorlet{urlcolor}{black}
\usepackage[
  bookmarks=true,
  breaklinks=true,
  pdfborder={0 0 0},
  citecolor=citecolor,
  linkcolor=linkcolor,
  urlcolor=urlcolor,
  colorlinks=true,
  linktocpage=false,
  hyperindex=true,
  colorlinks=true,
  linktocpage=false,
  linkbordercolor=white]{hyperref}

\usepackage{fontspec}
\usepackage[fixed]{fontawesome5}
\usepackage[babel=true]{microtype}
\defaultfontfeatures{Ligatures=TeX}
\setmainfont{Source Serif Pro}[
  Path           = fonts/source-serif-pro/SourceSerifPro-,
  Extension      = .otf,
  UprightFont    = Regular,
  ItalicFont     = LightIt,
  BoldFont       = Semibold,
  BoldItalicFont = It ]
\newfontfamily\RobotoSlab{Roboto Slab}[
  Path        = fonts/roboto-slab/RobotoSlab-,
  Extension   = .ttf,
  UprightFont = Light,
  BoldFont    = Regular ]

\definecolor{secbgcol}{HTML}{2D3E40}
\definecolor{highlight}{HTML}{36A8B2}
\definecolor{secftcol}{HTML}{ffffff}
\definecolor{doccol}{HTML}{4cbdc7}
\definecolor{myblue}{HTML}{12558c}

\usepackage{listings}
\definecolor{lightblue}{RGB}{240,245,255}
\definecolor{darkblue}{RGB}{40,40,85}
\lstset{
    language=Python,
    backgroundcolor = \color{lightblue},
    basicstyle=\tiny\ttfamily,
    breaklines=true,
    showstringspaces=false,
    keywordstyle=\color{green!35!black},
    commentstyle=\selectfont\color{gray},
    numberstyle=\color{blue},
    morekeywords={constexpr},
    classoffset=1,
    morekeywords={dense},
    keywordstyle=\color{myblue},
    classoffset=2,
    morekeywords={
        True, False, None, Undefined, import,
        and, or, in, is, not,
        as, if, else, elif, for,
        schema, mixin, protocol, check, assert,
        all, any, map, filter, lambda, rule
    },
    keywordstyle=\color{highlight},
    classoffset=3,
    morekeywords={place, f32, f64, u8, u16, u32, u64, i8, i16, i32, i64, f16, float, int},
    keywordstyle=\color{highlight},
%    moredelim=**[is][\color{red}]{@}{@},
    xleftmargin=0.1cm,
    xrightmargin=0.1cm,
    frame=tlbr,
    framesep=0.1cm,
    framerule=0pt,
}
\newcommand{\lcode}[1]{\lstinline[basicstyle=\scriptsize\ttfamily]{#1}}
\setlength\parindent{0pt}
\setlength{\tabcolsep}{2pt}
\baselineskip=0pt
\setlength\columnsep{0.5em}

\newcommand{\offset}{\vspace*{0.4em}}

\newcommand{\API}[1]{\tikz[baseline=(X.base)]
  \node [fill=doccol, rectangle, inner sep=2pt,rounded corners=1pt] (X)
        {\href{#1}{\color{white}{\tiny \sffamily \textbf{DOC}}}};}

\newcommand{\LINK}[2]{\tikz[baseline=(X.base)]
    \node [fill=doccol, rectangle, inner sep=2pt,rounded corners=1pt] (X)
        {\href{#2}{\color{white}{\tiny \sffamily \textbf{#1}}}};}

\newenvironment{myboxed}[1]
{\begin{mdframed}[linecolor=black,
                  backgroundcolor=white,
                  outerlinewidth=0.25pt,
                  %roundcorner=0.25em,
                  innertopmargin=1ex,
                  topline=true,
                  rightline=true,
                  leftline=true,
                  bottomline=true,
                  linecolor=black!0,
                  frametitleaboveskip=0.5em,
                  frametitlebelowskip=0.5em,
                  innerbottommargin=.5\baselineskip,
                  innerrightmargin=.5em,
                  innerleftmargin=.5em,
                  %userdefinedwidth=1\textwidth,
                  % frametitle={\scshape \bfseries \sffamily #1},
                  frametitle={\footnotesize\color{secftcol}\RobotoSlab \bfseries \hspace*{0mm} #1},
                  % frametitlerule=true,
                  %frametitlerulecolor=red,
                  frametitlebackgroundcolor=secbgcol,
                  frametitlerulewidth=2pt]}
{\end{mdframed}}

\begin{document}
\thispagestyle{empty}
% \footnotesize
\scriptsize

% -----------------------------------------------------------------------------

\begin{multicols*}{5}

% -----------------------------------------------------------------------------
% Header + Logo
% -----------------------------------------------------------------------------

\begin{overpic}[width=\columnwidth,tics=6, trim=-10 10 0 0, clip]{kcl-lang-logo.pdf}
\end{overpic}

\textbf{ }


%\vspace{\fill}


% -----------------------------------------------------------------------------
% Intro: https://kcl-lang.io/docs/user_docs/getting-started/intro
% -----------------------------------------------------------------------------

\begin{myboxed}{Intro \hfill \API{https://kcl-lang.io/docs/user_docs/getting-started/intro}}
\offset

\textbf{KCL} is an open\-source constraint-based record \& functional language mainly used in configuration and policy scenarios.

\textbf{ }

\href{https://github.com/kcl-lang/kcl}{\color{myblue}{\sffamily \textbf{https://github.com/kcl-lang/kcl}}}

\href{https://kcl-lang.io}{\color{myblue}{\sffamily \textbf{https://kcl-lang.io}}}

\end{myboxed}

% -----------------------------------------------------------------------------
% Installation
% -----------------------------------------------------------------------------

\begin{myboxed}{Installation \hfill \API{https://kcl-lang.io/docs/user_docs/getting-started/install}}

\hspace*{1mm}\raisebox{0.5pt}{\tiny$\bullet$}~\textbf{Docker}

\begin{lstlisting}
docker pull kcllang/kcl
\end{lstlisting}

\hspace*{1mm}\raisebox{0.5pt}{\tiny$\bullet$}~\textbf{macOS}

\begin{lstlisting}
brew install kcl-lang/tap/kcl
\end{lstlisting}

\hspace*{1mm}\raisebox{0.5pt}{\tiny$\bullet$}~\textbf{Linux}

\begin{lstlisting}
wget -q https://kcl-lang.io/script/install.sh -O - | /bin/bash
\end{lstlisting}

\hspace*{1mm}\raisebox{0.5pt}{\tiny$\bullet$}~\textbf{Windows}

\begin{lstlisting}
powershell -Command "iwr -useb https://kcl-lang.io/script/install.ps1 | iex"
\end{lstlisting}


\end{myboxed}

% -----------------------------------------------------------------------------
% Quick start
% -----------------------------------------------------------------------------

\begin{myboxed}{Quick start \hfill \API{https://kcl-lang.io/docs/user_docs/getting-started/kcl-quick-start}}
\begin{lstlisting}
# This is a KCL document

title = "KCL Example"

owner = {
    name = "The KCL Authors"
    data = "2020-01-02T03:04:05"
}

database = {
    enabled = True
    ports = [8000, 8001, 8002]
    data = [["delta", "phi"], [3.14]]
    temp_targets = {cpu = 79.5, case = 72.0}
}

servers = [
    {ip = "10.0.0.1", role = "frontend"}
    {ip = "10.0.0.2", role = "backend"}
]
\end{lstlisting}
\end{myboxed}


% -----------------------------------------------------------------------------
% VS Code
% -----------------------------------------------------------------------------

\begin{myboxed}{VS Code Extension \hfill \LINK{LINK}{https://marketplace.visualstudio.com/items?itemName=kcl.kcl-vscode-extension}}

\begin{overpic}[width=\columnwidth,tics=6, trim=-10 10 0 0, clip]{ide-vscode.png}
\end{overpic}

Highlighting, auto-completion, quick info hover and code navigation, etc

\end{myboxed}

% -----------------------------------------------------------------------------
\vspace{\fill}
% -----------------------------------------------------------------------------

% -----------------------------------------------------------------------------
% Keywords
% https://kcl-lang.io/docs/reference/lang/tour#keywords
% -----------------------------------------------------------------------------

\begin{myboxed}{Keywords \hfill \LINK{DOC}{https://kcl-lang.io/docs/reference/lang/tour}}
\offset

\begin{lstlisting}
True    False  None      Undefined  import
and     or     in        is         not
as      if     else      elif       for
schema  mixin  protocol  check      assert
all     any    map       filter     lambda
rule
\end{lstlisting}

\end{myboxed}

% -----------------------------------------------------------------------------
% Data Types
% -----------------------------------------------------------------------------

\begin{myboxed}{Data Types(bool/int/float/units/str) \hfill \LINK{DOC}{https://kcl-lang.io/docs/reference/lang/tour}}
\offset

\begin{lstlisting}
name = "Foo"  # exported, can only be set once.
_name = "Foo"  # internal and are mutable
$if = 3  # equal to `if = 2`
\end{lstlisting}

\textbf{boolean}

\begin{lstlisting}
a = True
b = False
\end{lstlisting}

\textbf{int}

\begin{lstlisting}
a = 123
b = 0x10 # hexadecimal literal
c = int("10") # int constructor
\end{lstlisting}


\textbf{float}

\begin{lstlisting}
a = 1.10
b = -35.59
c = 32.3e+18
d = 70.2E-12
e = float("112") # float constructor
\end{lstlisting}

\textbf{units}

\begin{lstlisting}
# SI
n = 1n  # 1e-09
u = 1u  # 1e-06
m = 1m  # 1e-03
k = 1k  # 1000
K = 1K  # 1000
M = 1M  # 1000000
G = 1G  # 1000000000
T = 1T  # 100000000000
P = 1P  # 1000000000000000
# IEC
Ki = 1Ki  # 1024
Mi = 1Mi  # 1024 ** 2
Gi = 1Gi  # 1024 ** 3
Ti = 1Ti  # 1024 ** 4
Pi = 1Pi  # 1024 ** 5
\end{lstlisting}

\textbf{str}

\begin{lstlisting}
'allows embedded "double" quotes'
"allows embedded 'single' quotes"

'''Three single quotes'''
"""Three double quotes"""

"""This is a long triple quoted string
may span multiple lines.
"""

s = "Hi\nHello"

# This is a KCL raw string with the `r` prefix.
raw_s = r"Hi\nHello"


x = 'The + operator ' + 'works, as well.'
x = str(3.5) # "3.5"

worldString = "world"
s = "Hello ${worldString}"

# This is a KCL raw string with the `r` prefix. 
raw_s = r"Hello ${worldString}"

x = "length"
assert len(x) == 6 # True
assert "{} {}".format("ab", "12") == 'ab 12'
\end{lstlisting}


\end{myboxed}

\vspace{\fill}

% -----------------------------------------------------------------------------

\begin{myboxed}{Data Types(List/Dict/Schema/Alias) \hfill \LINK{DOC}{https://kcl-lang.io/docs/reference/lang/tour}}
\offset

\textbf{list}

\begin{lstlisting}
list = [1, 2, 3]
assert len(list) == 3  # True
assert list[0] == 1  # True

list = [ _x for _x in range(10) if _x % 2 == 0]
assert list == [0, 2, 4, 6, 8] # True

# [1000, 2000, 3000]
dataLoop = [i if i > 2 else i + 1 for i in data]
\end{lstlisting}

\textbf{dict}

\begin{lstlisting}
a = {"one" = 1, "two" = 2, "three" = 3}
b = {'one' = 1, 'two' = 2, 'three' = 3}
assert a == b # True
assert len(a) == 3 # True

person = {
    base.count = 2
    base.value = "value"
    labels.key = base.value
}
\end{lstlisting}

\textbf{Schema}

\begin{lstlisting}
schema Person:
    firstName: str
    lastName: str
    age: int = 0 # default value: 0
\end{lstlisting}

\textbf{Type Alias}

\begin{lstlisting}
type Int = int
type String = str
type StringOrInt = String | Int
type IntList = [int]
type StringAnyDict = {str:}
\end{lstlisting}


\end{myboxed}

% -----------------------------------------------------------------------------

\begin{myboxed}{Operators \hfill}
\offset

\begin{lstlisting}
+   -   *   **  /   //  %
<<  >>  &   |   ^   <   >
~   <=  >=  ==  !=  @   \
\end{lstlisting}

\textbf{Arithmetic}

\begin{lstlisting}
assert 2 + 3 == 5
assert 2 - 3 == -1
assert 2 * 3 == 6
assert 5 / 2 == 2.5
assert 5 // 2 == 2
assert 5 % 2 == 1
\end{lstlisting}

\textbf{Equality and Relational}

\begin{lstlisting}
assert 2 == 2
assert 2 != 3
assert 3 > 2
assert 2 < 3
assert 3 >= 3
assert 2 <= 3
\end{lstlisting}

\textbf{Bitwise and Shift}

\begin{lstlisting}
value = 0x22
bitmask = 0x0f

assert (value & bitmask) == 0x02
assert (value & ~bitmask) == 0x20
assert (value | bitmask) == 0x2f
assert (value ^ bitmask) == 0x2d
assert (value << 4) == 0x220
assert (value >> 4) == 0x02
\end{lstlisting}

\textbf{Bitwise and Shift}

\begin{lstlisting}
_x = True and (col == 0 or col == 3)
\end{lstlisting}

\end{myboxed}

% -----------------------------------------------------------------------------

\begin{myboxed}{Operators \hfill}
\offset

\begin{lstlisting}
empty_String = ""
empty_String is not None # True

1 in [1, 2, 3]                  # True

d = {one = 1, two = 2}
"one" in d                      # True
"three" in d                    # False
1 in d                          # False
[] in d                         # False

"nasty" in "dynasty"            # True
"a" in "banana"                 # True
"f" not in "way"                # True

# Data is a schema with attributes one and two
d = Data {one = 1, two = 2}

"one" in d                      # True
"three" in d                    # False

\end{lstlisting}

\end{myboxed}

% -----------------------------------------------------------------------------

\begin{myboxed}{Control Flow Statements \hfill}
\offset

\textbf{If}

\begin{lstlisting}
a = 10
if a == 0:
    print("a is zero")
elif a < 100:
    print("a < 100")
else:
    print("a >= 100")

_result = "success" if success else "failed"
\end{lstlisting}

\textbf{Assert}

\begin{lstlisting}
assert a != b
assert a == b, "SOS"
\end{lstlisting}

\textbf{Function}

\begin{lstlisting}
func = lambda x: int, y: int -> int {
    x + y
}
a = func(1, 1)  # 2
\end{lstlisting}

\end{myboxed}

\vspace{\fill}

% -----------------------------------------------------------------------------
% Top-Level Argument
% -----------------------------------------------------------------------------

\begin{myboxed}{Top-Level Argument \hfill}
\offset

\begin{lstlisting}
# kcl -DbankCard=123 employee.k
bankCard = option("bankCard")

# kcl main.k -D list_key='[1,2,3]' -D dict_key='{"key":"value"}'
list_key = option("list_key")
dict_key = option("dict_key")
\end{lstlisting}

\textbf{Arguments with Setting Files}

\begin{lstlisting}
# kcl -Y setting.yaml employee.k

# setting.yaml
kcl_options:
  - key: key_number
    value: 1
  - key: key_dict
    value:
      innerDictKey:  innerDictValue
  - key: key_list
    value:
      - 1
      - 2
      - 3
  - key: bankCard
    value: 123
\end{lstlisting}


\end{myboxed}

\vspace{\fill}


% -----------------------------------------------------------------------------
% Concepts
% -----------------------------------------------------------------------------

\begin{myboxed}{Concepts \hfill \API{https://kcl-lang.io/docs/user_docs/getting-started/intro}}
\offset

\textbf{KCL: Config}

\begin{lstlisting}
import kubernetes.core.v1

# Create a kubernetes deployment resource
deployment = v1.Deployment {
    metadata.name = "nginx"
    metadata.labels.app = metadata.name
    spec = {
        replicas = 3
        selector.matchLabels.app = metadata.name
        template = {
            metadata.labels.app = metadata.name
            spec.containers = [{
                name = metadata.name
                image = "nginx:1.14.2"
                ports = [{containerPort=80}]
            }]
        }
    }
}
\end{lstlisting}


\textbf{KCL: Schema}

\begin{lstlisting}
import units

type UnitType = units.NumberMultiplier

# Define a schema named Resource with
# three attributes and constraints
schema Resource:
    cpu: int | UnitType = 1
    memory: UnitType = 1024Mi
    dist: UnitType = 10Gi

    check:
        0 < cpu < 64
        0 < memory <= 64 Gi
        0 < disk <= 1Ti
\end{lstlisting}

\textbf{KCL: Rule}

\begin{lstlisting}
data: Data = option("data")
input: Input = option("input")

# Define a RBAC rule
rule Allow:
    any grant in UserIsGranted() {
        input.action == grant.action and input.type == grant.type
    }
    any user in data.user_roles[input.user] {
        user == "admin"
    }

rule UserIsGranted:
    [   grant 
        for role in data.user_roles[input.user]
        for grant in data.role_grants[role]
    ]

allow = Allow or False
\end{lstlisting}

\textbf{KCL: Lambda}

\begin{lstlisting}
# Transform input resource  and change annotation

transformer = lambda res {
    res | {
        metadata.annotations: {
            "managed-by" = "kcl"
        }
    } if res.kind == "Deployment" else res
}

output = [transformer(res) for res in option("input")]
\end{lstlisting}



\textbf{KCL = Config + Schema + Rule + Lambda}

\end{myboxed}

\vspace{\fill}

% -----------------------------------------------------------------------------
% end
% -----------------------------------------------------------------------------

\end{multicols*}
\end{document}
